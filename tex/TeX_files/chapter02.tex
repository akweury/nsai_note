\chapter{Dataset}

\section{CLEVR}
CLEVR\cite{CLEVR} is a dataset for \textit{image reasoning}. CLEVRER is a dataset for \textit{video reasoning}.

\paragraph{Objects and Relationships}
\begin{itemize}
	\item Three object shapes: cube, sphere, and cylinder.
	\item Two absolute sizes: small and large.
	\item Two materials: shiny and matte. 
	\item Eight colors. 
	\item Four relationships: left, right, behind, in front
\end{itemize}

\paragraph{Scene representation}
A scene can be represented by a scene graph.

\paragraph{Image generation}
Images are generated by randomly sampling a scene graph and render it using Blender.

\paragraph{Question representation}
Each question is associated with a \textit{functional program} that can be executed on an image's scene graph.

Functional programs are built from simple basic building blocks, which are elementary operations of visual reasoning, such as querying object attributes, counting sets of objects, or comparing values.

Question size is defined as the number of functions in its program. Effective question is defined as the smallest equivalent program. Effective size is the size of effective question. Large effective size leads to long reasoning chains, which gives a higher error rate.

Complex questions can be represented by compositions of simple building blocks.

\paragraph{Question families}
A question family contains a template for constructing functional programs and several text templates providing multiple ways of expressing these programs in natural language.


\paragraph{Question generation}
\begin{enumerate}
	\item choose a question family (depth-first search)
	\item select values for each of its template parameters
	\item execute the resulting program on the image's scene graph to find the answer
	\item use one of the text templates from the question family to generate the final natural-language question
\end{enumerate}


\paragraph{Question topology}
\begin{enumerate}
	\item chain-structured, \textit{what color is the cube to the right of the yellow sphere}.
	\item tree-structured, maybe more difficult, require models to perform two sub-tasks in parallel before fusing their results. \textit{How many cylinders are in front of the small thing and on the left side of the green object.}
\end{enumerate}






\section{DAQUAR}


This dataset is proposed in \cite{Daquar}. 


\begin{itemize}
	\item 6795 training questions on 795 images
	\item 5673 test questions on 654 images
	\item question types
	\begin{itemize}
		\item object
		\item color
		\item number
	\end{itemize}
\end{itemize}


\section{COCO-QA}
This dataset is proposed in \cite{coco-qa}.
\begin{itemize}
	\item 78736 training questions on 8000 images
	\item 38948 test questions on 4000 images
	\item question types
	\begin{itemize}
		\item object
		\item color
		\item number
		\item location
	\end{itemize}
\end{itemize}

\section{VQA}
This dataset is proposed in \cite{VQA}.

\begin{itemize}
	\item 0.25M images
	\item  0.76M questions
	\item  10M answers
\end{itemize}



\section{Image Net}
\cite{Image-Net}











