\chapter{The First Chapter}



\section{Main Task}
Es müssen mathematische und algorithmische Grundlagen für neurosymbolische KI-Systeme entwickelt werden.
Ein ganzheitlicher Ansatz sollte die Perspektive der Verwendung verallgemeinerbarer und robuster neuronaler Netzmodelle mit der Ausdrucks- und Argumentationskraft symbolischer Systeme im Entwurfsprozess verbinden. Zu diesem Zweck sollen neuartige neurosymbolische Modelle und dynamische Architekturen entwickelt werden, um in Gegenwart von strukturierten, verrauschten und heterogenen Daten effektiv zu lernen.
The goal of neural-symbolic AI(NeSy) is to integrate symbolic reasoning and neural networks, where the first topic focus on the task of \textbf{reasoning} and the second topic focus on the task of \textbf{learning}. 
NeSy is different from statistical relational learning and artificial intelligence (StarIAI),
which is also a domain that focus on the integration of learning and reasoning.\cite{Susskind2021} talked about neural symbolic AI.
Image question answering often requires multiple steps of reasoning.\cite{SAN}


\section{Related Work}
Paper\cite{yang22} injects discrete logical constraints(DLC) into neural network.
It represents the DLC as a loss function, then use a mechanism called \textit{Straight-Through-Estimator(STE)} to update the neural network's weight in the direction that the binarized outputs satisfy the logical constraints. They argued that this method is better than the others that require heavy symbolic computation for computing gradients.



